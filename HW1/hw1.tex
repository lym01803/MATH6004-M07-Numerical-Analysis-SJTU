\documentclass{article}
\usepackage{ctex}
\usepackage{amsmath,amscd,amsbsy,amssymb,latexsym,url,bm,amsthm}
\usepackage{epsfig,graphicx,subfigure}
\usepackage{enumitem,balance,mathtools}
\usepackage{wrapfig}
\usepackage{mathrsfs, euscript}
\usepackage[usenames]{xcolor}
\usepackage{hyperref}
\usepackage{caption}
\usepackage{setspace}
%\usepackage{subcaption}
\usepackage{float}
\usepackage{listings}
%\usepackage{enumerate}
%\usepackage{algorithm}
%\usepackage{algorithmic}
%\usepackage[vlined,ruled,commentsnumbered,linesnumbered]{algorithm2e}
\usepackage[ruled,lined,boxed,linesnumbered]{algorithm2e}
\usepackage{tikz}

\newtheorem{theorem}{Theorem}[section]
\newtheorem{lemma}[theorem]{Lemma}
\newtheorem{proposition}[theorem]{Proposition}
\newtheorem{corollary}[theorem]{Corollary}
\newtheorem{exercise}{Exercise}[section]
\newtheorem*{solution}{Solution}

\renewcommand{\thefootnote}{\fnsymbol{footnote}}
\renewenvironment{solution}[1][Solution]{~\\ \textbf{#1.}}{~\\}

\newcommand{\prob}{\mathtt{Pr}}

\newcommand{\postscript}[2]
    {\setlength{\epsfxsize}{#2\hsize}
    \centerline{\epsfbox{#1}}}

\renewcommand{\baselinestretch}{1.0}
\SetKwFor{Function}{function}{:}{end}
\setlength{\oddsidemargin}{-0.365in}
\setlength{\evensidemargin}{-0.365in}
\setlength{\topmargin}{-0.3in}
\setlength{\headheight}{0in}
\setlength{\headsep}{0in}
\setlength{\textheight}{10.1in}
\setlength{\textwidth}{7in}

\title{计算方法 作业1}
\author{刘彦铭\ \ 学号:122033910081}
\date{编辑日期:\ \today}

\begin{document}

\maketitle

李庆杨等, 数值分析, 第5版, 华中科大, P.12, 1,2,4,5,10,11,13,14

\begin{enumerate}
    \item Page12 习题1
    
    $\ln (x^*) - \ln (x) = \ln \left(1 + \dfrac{x^*-x}{x}\right) = \ln (1 + \delta) = \ln (1) + \delta + O(\delta^2) $.
    忽略二阶以上的项,误差为 $\delta$ .

    \item Page12 习题2
    
    $\dfrac{(x^*)^n - x^n}{x^n} = \left(\dfrac{x^*}{x}\right)^n - 1 = \left(1 + \dfrac{x^*-x}{x}\right)^n - 1$.
    记 $\delta = \dfrac{x^*-x}{x} = 2\%$, 有:
    
    $(1+\delta)^n - 1 = n\cdot \delta + O(\delta ^ 2)$ 忽略二阶以上的项,可得相对误差为 $n\cdot\delta$, 即 $0.02n$ .

    \item Page12 习题4
    
    \begin{enumerate}
        \item [(1)] $|e^*(x_1^*+x_2^*+x_4^*)| \leq  |e_1^*| + |e_2^*| + |e_4^*| \leq 0.5\times 10^{-4} + 0.5 \times 10 ^{-3} + 0.5 \times 10 ^{-3} = 1.05\times 10^{-3}$
        \item [(2)] $|e^*(x_1^*x_2^*x_3^*)| \leq |e_1^*x_2^*x_3^*| + |x_1^*e_2^*x_3^*| + |x_1^*x_2^*e_3^*| \leq 0.2148$
        \item [(3)] $\left|e^*\left(\dfrac{x_2^*}{x_4^*}\right)\right| \leq \left|\dfrac{e_2^*}{x_4^*}\right| + \left|\dfrac{x_2^*e_4^*}{(x_4^*)^2}\right| \leq 8.866 \times 10^{-6}$
    \end{enumerate}

    \item Page12 习题5

    $\dfrac{V^*-V}{V} = \dfrac{(R^*)^3 - R^3}{R^3} = \left(\dfrac{R^*-R}{R} + 1\right) ^ 3 - 1 = 3\cdot\dfrac{R^*-R}{R} + O\left(\left(\dfrac{R^*-R}{R}\right)^2\right)$

    忽略二阶以上的项,所以半径的相对误差限应该为 $|\dfrac{R^*-R}{R}|\leq \dfrac{1}{3}\times 1\% = 0.33\%$

    \item Page12 习题10 
    
    绝对误差 $s^* - s = g\cdot t\cdot e_t^* + O((e_t^*)^2)$. 绝对误差关于 $t$ 递增;

    相对误差 $\dfrac{s^* - s}{s} = \dfrac{g}{t}\cdot e_t^* + O((e_t^*)^2)$. 相对误差关于 $t$ 递减。

    \item Page12 习题11 
    
    $y_n = 10 y_{n-1} - 1 \Rightarrow e_n = 10 e_{n-1} \Rightarrow e_{10} = 10 ^ {10} \cdot e_{0} = 1.41\times 10 ^ {10}$. 该计算过程不稳定。

    \item Page12 习题13
    
    $f(30) \approx -4.094 $

    考虑 $f(x) = \ln(x - \sqrt{x^2 - 1})$ 求 $f(30)$ 时的误差, 由于$30$是整数,在常见计算机系统上没有浮点误差,故而误差来源于 $\sqrt{x^2-1}$ 的开平方操作。设这一误差为 $e$ .

    则计算 $f(30)$ 时的误差估计为 $\dfrac{e}{x-\sqrt{x^2-1}} = (x + \sqrt{x^2-1})e \approx 60 e$.

    而使用 $-\ln(x+\sqrt{x^2-1})$ 计算 $f(30)$ 时的误差估计为 $\dfrac{e}{x+\sqrt{x^2-1}} \approx \dfrac{e}{60}$.


    \item Page12 习题14
    
    解为 $\left\{\begin{array}{l} x_1 = \dfrac{10^{10}}{10^{10} - 1} \\ x_2 = \dfrac{10^{10} - 2}{10^{10} - 1} \end{array}\right.$
    假定只用三位数计算,消元中计算 $10^{10}-1$, $10^{10}-2$ 时会对阶,得到 $10^{10}$ 导致``大数吃掉小数'',从而得到
    $x_1=x_2=1$的解。虽然在本例中解的误差很小,但这种计算不可靠。

\end{enumerate}

\end{document}
