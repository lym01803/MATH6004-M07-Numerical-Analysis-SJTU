\documentclass{article}
\usepackage{ctex}
\usepackage{amsmath,amscd,amsbsy,amssymb,latexsym,url,bm,amsthm}
\usepackage{epsfig,graphicx,subfigure}
\usepackage{enumitem,balance,mathtools}
\usepackage{wrapfig}
\usepackage{mathrsfs, euscript}
\usepackage[usenames]{xcolor}
\usepackage{hyperref}
\usepackage{caption}
\usepackage{setspace}
%\usepackage{subcaption}
\usepackage{float}
\usepackage{listings}
%\usepackage{enumerate}
%\usepackage{algorithm}
%\usepackage{algorithmic}
%\usepackage[vlined,ruled,commentsnumbered,linesnumbered]{algorithm2e}
\usepackage[ruled,lined,boxed,linesnumbered]{algorithm2e}
\usepackage{tikz}

\newtheorem{theorem}{Theorem}[section]
\newtheorem{lemma}[theorem]{Lemma}
\newtheorem{proposition}[theorem]{Proposition}
\newtheorem{corollary}[theorem]{Corollary}
\newtheorem{exercise}{Exercise}[section]
\newtheorem*{solution}{Solution}

\renewcommand{\thefootnote}{\fnsymbol{footnote}}
\renewenvironment{solution}[1][Solution]{~\\ \textbf{#1.}}{~\\}

\newcommand{\prob}{\mathtt{Pr}}

\newcommand{\postscript}[2]
{\setlength{\epsfxsize}{#2\hsize}
\centerline{\epsfbox{#1}}}

\renewcommand{\baselinestretch}{1.0}
\SetKwFor{Function}{function}{:}{end}
\setlength{\oddsidemargin}{-0.365in}
\setlength{\evensidemargin}{-0.365in}
\setlength{\topmargin}{-0.3in}
\setlength{\headheight}{0in}
\setlength{\headsep}{0in}
\setlength{\textheight}{10.1in}
\setlength{\textwidth}{7in}

\title{计算方法\ 作业4}
\author{刘彦铭\ \ ID: 122033910081}
\date{Last Edited:\ \today}

\begin{document}

\maketitle

李庆杨等, 数值分析, 第5版, 华中科大, P.217,
1,2,3,4,5(1),7,8,9,11,12,13,14,15,16,18,19

\begin{spacing}{1.5}

\begin{itemize}
    \item [1.] 习题1
    
    $A =\left[\begin{array}{ccc}5&2&1\\-1&4&2\\2&-3&10\end{array}\right], b=\left[-12,20,3\right]^\top$, $Ax = b$. Jacobi迭代: $x \leftarrow Bx + f$, 其中 
    
    $B = D^{-1}(L+U) = \left[\begin{array}{ccc}
        0 & -0.4 & -0.2 \\ 0.25 &   0 & -0.5 \\ -0.2 & 0.3 &   0
    \end{array}\right]$,
    $f = D^{-1}b = \left[\begin{array}{c}
        -2.4 \\   5 \\ 0.3
    \end{array}\right]$.

    Gauss-Seidel迭代: $x \leftarrow Gx + g$, 其中 
    
    $G = (D-L)^{-1}U = \left[\begin{array}{ccc}
        0 & -0.4 & -0.2 \\   0 & -0.1 & -0.55 \\   0 & 0.05 & -0.125
    \end{array}\right], 
    g = (D-L)^{-1}b=\left[\begin{array}{c}
        -2.4 \\ 4.4 \\ 2.1
    \end{array}\right]$.

    \begin{itemize}
        \item [(1)] 计算得到 $\rho(B) = 0.506 < 1, \rho(G) = 0.200 < 1$, 所以 Jacobi迭代 和 Gauss-Seidel迭代 在本例中都是收敛的。
        \item [(2)] 两种迭代方法都能得到 $x =   \left[\begin{array}{c}
            -4.0000 \\ 3.0000 \\   2.0000
        \end{array}\right]$ 的解

    \end{itemize}

    \item [2.] 习题2
    
    $A=\left[\begin{array}{cc}0&0\\2&0\end{array}\right]$ 是幂零矩阵,$A^k=0, \forall k\geq 2$. 所以 $\forall k\geq 2, I + A + A^2 + \cdots + A^k = I + A$, 该级数收敛

    \item [3.] 习题3
    
    不妨设 $A$ 为 $n$ 阶方阵, $m = \max_{ij} |A_{ij}|$, 下用数学归纳法证明: $\max_{ij} |(A^k)_{ij}| \leq n^{k-1}m^k$.
    \begin{itemize}
        \item [(1)] $k=1$ 时显然成立;
        \item [(2)] $\forall 1\leq i,j\leq n$, $|(A^{k+1})_{ij}| = |\sum_k (A^k)_{ik}A_{kj}| \leq n \cdot (n^{k-1}m^k \cdot m) = n^km^{k+1}$
    \end{itemize}

    所以存在常数 $N = \lceil 2nm \rceil$, 当 $k > N$ 时, $\max_{ij} |(A^{k}/k!)_{ij}| \leq \dfrac{n^{N-1}m^N}{N!}\cdot \left(\dfrac{1}{2}\right)^{k-N}$, 从而有 
    
    $\lim_{k\to\infty} \max_{ij} |(A^k/k!)_{ij}| = 0$.
    由于 $|P|_\infty \leq n\cdot \max_{ij} |P_{ij}|, \forall P\in\mathbb{R}^{n\times n}$, 所以 $\lim_{k\to\infty} |A^k/k!|_\infty = 0$, 即该序列收敛于零。

    \item [4.] 习题4
    
    写作矩阵形式 $\left[\begin{array}{c}x_1\\x_2\end{array}\right]\leftarrow 
    \left[\begin{array}{cc}
         & -a_{12}/a_{11}\\
        -a_{21}/a_{22} &  
    \end{array}\right]
    \left[\begin{array}{c}x_1\\x_2\end{array}\right]
    +\left[\begin{array}{c}b_1/a_{11}\\b_2/a_{22}\end{array}\right] = Bx + f$

    收敛当且仅当 $\rho(B) < 1$. 考虑到 $\det (xI - B) = x^2 - (a_{12}a_{21}) / (a_{11}a_{22})$, 故 $\rho(B) = \sqrt{|a_{12}a_{21}|/|a_{11}a_{22}|}$. 由此得 收敛的充要条件是 $\left|\dfrac{a_{12}a_{21}}{a_{11}a_{22}}\right| < 1$.

    \item [5.] 习题5(1)
    
    Jacobi迭代:
    $D^{-1}(L+U) = \left[\begin{array}{ccc}
        0 & -0.4 & -0.4 \\ -0.4 &   0 & -0.8 \\ -0.4 & -0.8 &   0
    \end{array}\right]$,
    $\rho(D^{-1}(L+U)) = 1.093 > 1$, 故不收敛;

    Gauss-Seidel迭代:
    $(D-L)^{-1}U =  \left[\begin{array}{ccc}
        0 & -0.4 & -0.4 \\   0 & 0.16 & -0.64 \\   0 & 0.032 & 0.672
    \end{array}\right]$,
    $\rho((D-L)^{-1}U) = 0.628 < 1$, 故收敛.

    \item [6.] 习题7
    
    $Ax = b$, 其中 $A$ 对称正定。Jacobi 迭代有 $x^{(k+1)} = Bx^{(k)} + f$ 其中 $B = D^{-1}(L+U) = I - D^{-1}A$. 若 $\lambda$ 是 $D^{-1}A$ 的特征值,那么 $1-\lambda$ 是 $B = I - D^{-1}A$ 的特征值。就 5(1) 中的例子而言, $A$ 是对称正定矩阵, $D^{-1}A$ 的一个特征值为 $2.093$, 故 $-1.093$ 是 $B$ 的一个特征值,从而 $\rho(B)\geq 1.093 > 1$, 迭代不收敛。 

    \item [7.] 习题8
    
    $Ax = b$, $A = \left[\begin{array}{cccc}
        1 &    0 & -0.25 & -0.25 \\    0 &    1 & -0.25 & -0.25 \\ -0.25 & -0.25 &    1 &    0 \\ -0.25 & -0.25 &    0 &    1
    \end{array}\right]$, $b = \left[\begin{array}{c}
        0.5 \\  0.5 \\  0.5 \\  0.5
    \end{array}\right]$, 
    \begin{itemize}
        \item [(1)]  Jacobi迭代: $x\leftarrow B_0x + f$, 
        $B_0 = \left[\begin{array}{cccc}
            0 &    0 & 0.25 & 0.25 \\    0 &    0 & 0.25 & 0.25 \\ 0.25 & 0.25 &    0 &    0 \\ 0.25 & 0.25 &    0 &    0
        \end{array}\right]$, $f = \left[\begin{array}{c}
            0.5 \\  0.5 \\  0.5 \\  0.5
        \end{array}\right]$. 计算得到 $\det(xI - B_0) = x^2(x^2-1/4)$, 故 $\rho(B_0) = 1/2 < 1$.
        \item [(2)] Gauss-Seidel迭代: $B_0 = (D - L)^{-1} U$, $B_0 = \left[\begin{array}{cccc}
            0 &    0 & 0.25 & 0.25 \\    0 &    0 & 0.25 & 0.25 \\    0 &    0 & 0.125 & 0.125 \\    0 &    0 & 0.125 & 0.125
        \end{array}\right]$. 计算得到 $\det(xI - B_0) = x^3(x-1/4)$, 故 $\rho(B_0) = 1/4 < 1$.
        \item [(3)] 由 (1) (2) 知,改方程组的 Jacobi迭代 与 Gauss-Seidel迭代 都收敛.
    \end{itemize}

    \item [8.] 习题9
    
    矩阵形式的迭代公式 $x \leftarrow (D-\omega L)^{-1}((1-\omega)D + \omega U)x + \omega(D-\omega L)^{-1}b$.

    取 $x_0 = [0, 0, 0]^\top$, 编程计算可得:解为 $x = [0.50000, 1.00000, -0.50000]$. 当 $\omega$ 取 $1.03, 1, 1.1$ 时, 达到题设精度要求的迭代次数分别为 $5,6,6$ 次。

    \item [9.] 习题11
    
    整理得到迭代公式 $x\leftarrow (I-\omega A)x + \omega b$, 迭代收敛当且仅当 $\rho(I-\omega A) < 1$.

    由于 $A$ 对称正定,故存在正交矩阵 $Q$ 使得 $A = Q \Lambda Q^\top$, 其中 $\Lambda = \mathtt{diag}(\lambda_1,\cdots,\lambda_n)$, $\lambda_i$ 是矩阵 $A$ 的特征值,且均为正实数。故 $I - \omega A = QQ^\top - \omega Q\Lambda Q^\top = Q(I - \omega\Lambda)Q^\top$. 所以对于 $I-\omega A$ 的任意特征值 $\gamma$, 存在 $A$ 的特征值 $\lambda$, 使得 $\gamma = 1-\omega\lambda$. 当 $0< \omega < 2/\beta$,$0<\alpha\leq \lambda(A)\leq \beta$ 时有 $-1< 1-\omega\beta\leq\gamma = 1 - \omega\lambda \leq 1-\alpha\omega < 1$, 此时 $\rho(I-\omega A) < 1$.

    \item [10.] 习题12
    
    \begin{itemize}
        \item [(1)] 根据 Gauss-Seidel迭代的公式有: $x^{(k+1)}_i = \dfrac{1}{a_{ii}}\cdot\left(-\sum_{j<i} a_{ij}x^{(k+1)}_j - \sum_{j>i}a_{ij}x^{(k)}_{j} + b_i\right)$, 所以 
        
        $x^{(k+1)}_i = \dfrac{1}{a_{ii}}\cdot\left(-\sum_{j<i} a_{ij}x^{(k+1)}_j - \sum_{j\geq i}a_{ij}x^{(k)}_{j} + a_{ii}x^{(k)}_{i} + b_i\right) = x_{i}^{(k)} + r_i^{(k+1)}/a_{ii}$.

        \item [(2)] 根据题中定义有 $r_i^{(k+1)} = b_i - \sum_{j<i} a_{ij}x^{(k+1)}_j - \sum_{j\geq i} a_{ij}x^{(k)}_j = \sum_j a_{ij}x^*_{j} - \sum_{j<i} a_{ij}x^{(k+1)}_j - \sum_{j\geq i} a_{ij}x^{(k)}_j = \sum_{j<i} a_{ij}\epsilon_j^{(k+1)} +\sum_{j\geq i} a_{ij}\epsilon_j^{(k)}$, 其中 $\epsilon^{(k)} := x^* - x^{(k)}$. (题目中定义的符号应该反了)。 由(1)有 
        
        $\epsilon_i^{(k+1)} = x_i^* - x^{(k+1)}_i = x_i^* - x_i^{(k)} - r_i^{(k+1)}/a_{ii} = \epsilon_i^{(k)} - r^{(k+1)}_i/a_{ii}$. 

        \item [(3)] 为避免繁琐的上标,这里用 $r$ 表示 $r^{(k+1)}$, 用 $\epsilon,\epsilon^\prime$ 分别表示 $\epsilon^{(k)},\epsilon^{(k+1)}$. 由于 $A$ 对称,故而 $A = D - L - U = D - L - L^\top$. 将 (2) 中结论写作向量形式,有
        $\left\{\begin{array}{l}\epsilon^\prime = \epsilon - D^{-1}r\\r = -L\epsilon^\prime + D\epsilon - L^\top \epsilon\end{array}\right.$, 消去 $\epsilon^\prime$ 整理得到 $A\epsilon = (I - LD^{-1})r = (D - L)D^{-1}r$. 于是有 

        $$\begin{array}{ll}
        Q(\epsilon^\prime) - Q(\epsilon) &= (\epsilon - D^{-1}r)^\top A (\epsilon - D^{-1}r) - \epsilon^\top A \epsilon\\
        &= -2r^\top D^{-1}A\epsilon  + r^\top D^{-1} A D^{-1} r\\
        &= -2r^\top D^{-1}(D - L)D^{-1}r + r^\top D^{-1} (D - L - L^\top) D^{-1} r\\
        &= r^\top D^{-1} (-D + L - L^\top) D^{-1} r\\
        &= -r^\top D^{-1} r + r^\top D^{-1} L D^{-1} r - r^\top D^{-1} L^\top D^{-1} r
        \end{array}$$

        注意到 $r^\top D^{-1} L D^{-1} r\in\mathbb{R}^{1\times 1}$, 所以 $r^\top D^{-1} L D^{-1} r = (r^\top D^{-1} L D^{-1} r)^\top = r^\top D^{-1} L^\top D^{-1} r$. 所以有 $Q(\epsilon^\prime) - Q(\epsilon) = -r^\top D^{-1} r = - \sum_{j=1}^n \dfrac{\left(r^{(k+1)}_j\right)^2}{a_{jj}}$ .

        \item [(4)] (或许应当限定 $A$ 是对称矩阵?) 
        
        根据(3) 有 $\forall k, Q(\epsilon^{(k+1)}) \leq Q(\epsilon^{(k)})$, 所以 $\forall k, Q(\epsilon^{(0)}) \geq Q(\epsilon^{(k)})$.

        对于任意的 $y\in\mathbb{R}^{n\times 1}$, 取 $x_0 = y$ 为初始值, 利用 Gauss-Seidel迭代 解方程 $Ax = 0$. 由于 $A$ 非奇异, 所以 有理论上的唯一解 $x^* = 0$. 由于对于任意初始值都收敛,所以当取定 $x_0 = y$ 时, 迭代收敛于 $x^*=0$. 所以 $y^\top A y = (x_0 - x^*)^\top A (x_0 - x^*) = Q(\epsilon^{(0)}) \geq \lim_{k\to\infty} Q(\epsilon^{(k)}) = 0$. 可以验证当且仅当 $x_0 = y = 0$ 时取得等号,所以 $A$ 正定。
        
    \end{itemize}

    \item [11.] 习题13
    
    写作矩阵形式: $\left[\begin{array}{cc}A&B\\B&A\end{array}\right]\left[\begin{array}{c}z_1\\z_2\end{array}\right]=\left[\begin{array}{c}b_1\\b_2\end{array}\right]$
    \begin{itemize}
        \item [(1)] $\left[\begin{array}{c}z_1\\z_2\end{array}\right]\leftarrow \left[\begin{array}{cc}&-A^{-1}B\\-A^{-1}B\end{array}\right]\left[\begin{array}{c}z_1\\z_2\end{array}\right] + \left[\begin{array}{c}A^{-1}b_1\\A^{-1}b_2\end{array}\right]$.
        注意到 $\det\left(\left[\begin{array}{cc}xI&-A^{-1}B\\-A^{-1}B&xI\end{array}\right]\right) = \det(xI + A^{-1}B)\det(xI - A^{-1}B)$, 所以收敛当且仅当 $\rho\left(\left[\begin{array}{cc}&-A^{-1}B\\-A^{-1}B\end{array}\right]\right) = \rho (A^{-1}B) < 1$.
        \item [(2)] $\left[\begin{array}{c}z_1\\z_2\end{array}\right]^{(m+1)} = \left[\begin{array}{cc}0&-A^{-1}B\\0&0\end{array}\right]\left[\begin{array}{c}z_1\\z_2\end{array}\right]^{(m)} + \left[\begin{array}{cc}0&0\\-A^{-1}B&0\end{array}\right]\left[\begin{array}{c}z_1\\z_2\end{array}\right]^{(m+1)} + \left[\begin{array}{c}A^{-1}b_1\\A^{-1}b\end{array}\right]$, 
        
        即 $\left[\begin{array}{c}z_1\\z_2\end{array}\right]\leftarrow \left[\begin{array}{cc}I&0\\-A^{-1}B&I\end{array}\right]\left[\begin{array}{cc}0&-A^{-1}B\\0&0\end{array}\right]\left[\begin{array}{c}z_1\\z_2\end{array}\right] + \left[\begin{array}{cc}I&0\\-A^{-1}B&I\end{array}\right]\left[\begin{array}{c}A^{-1}b_1\\A^{-1}b\end{array}\right]$.

        记作 $z \leftarrow Gz + f$, $G = \left[\begin{array}{cc}0&-A^{-1}B\\0&A^{-1}BA^{-1}B\end{array}\right]$, $\det(G) = x^n\cdot\det(xI - A^{-1}BA^{-1}B)$. 所以迭代收敛当且仅当 $\rho(G) = \rho (A^{-1}BA^{-1}B) < 1$.

        考虑$A^{-1}B$的Schur标准型 $A^{-1}B = UR U^\star$, 有 $A^{-1}BA^{-1}B = UR^2U^\star$. 其中$R$ 是上三角阵,$R^2$ 也是上三角阵且对角线元素是 $R$ 中对应元素的平方。所以 $A^{-1}B$ 的非零特征值 $\lambda$ 与 $A^{-1}BA^{-1}B$ 的非零特征值 $\lambda^2$ 对应。所以迭代收敛当且仅当 $\rho(G) = \rho(A^{-1}BA^{-1}B) = \rho(A^{-1}B)^2 < 1$, 即 $\rho(A^{-1}B)<1$.
    \end{itemize}
    根据上述推导可知,两种迭代方法同时收敛(或者不收敛)。当收敛时, 由于 $\rho(A^{-1}B)^2 < \rho(A^{-1}B) < 1$, 故第二种方法的收敛速度更快.

    \item [12.] 习题14
    
    $\det(xI - A) = (x - 1 + a)^2 (x - 1 - 2a)$, 故 $A$ 的特征值为 $1 - a, 1 - a, 1 + 2a$. 所以对于 $-1/2 < a < 1$, $A$ 的全部特征值为正实数,$A$ 正定. Jacobi迭代收敛,当且仅当 $\rho(D^{-1}(L+U)) = \rho \left(\left[\begin{array}{ccc}0&-a&-a\\-a&0&-a\\-a&-a&0\end{array}\right]\right)$. 其特征值为 $-a, -a ,2a$, 故要保证 $\rho < 1$ 需要有 $-1/2 < a < 1/2$.

    \item [13.] 习题15
    
    取 $P = \left[\begin{array}{cccc}1&0&0&0\\0&0&1&0\\0&1&0&0\\0&0&0&1\end{array}\right]$, 可以验证 $PAP^\top$ 是形如 $\left[\begin{array}{cc}A_{11}&A_{12}\\0&A_{22}\end{array}\right]$ 的矩阵。

    \item [14.] 习题16
    
    由于 $C$ 的特征值全部为 $0$, 考虑 $C$ 的Schur标准型, $C = URU^\star$, $R$ 是对角线全 $0$ 的上三角阵。容易验证:$\forall k\geq n, R^k = 0$, $C^k = UR^n U^\star = 0$. 因为 $x^{(k)} = C^{k}x^{(0)} + \sum_{0\leq i < k} C^{i}g$, 所以 $\forall k\geq n$, $x^{(k)} = \sum_{0\leq i < n} C^{i}g$, 即最多迭代 $n$ 次就收敛。

    补充对 $R^n=0$ 的证明:下归纳证明 $(R^{k})_{ij} = 0, \forall i < j + k$ : (1) 对于 $k=1$ 成立;(2) $(R^k)_{ij} = 0, \forall i < j + k$ $\Rightarrow$ $(R^{k+1})_{ij} = (R^k R)_{ij} = \sum_t (R^k)_{it} R_{tj}$ . 当 $i < j + k + 1$ 时, $i \geq t + k$ 与 $t \geq j + 1$ 不能同时成立,故 $(R^k)_{it}$ 与 $R_{tj}$ 中至少有一个为 $0$, 从而推出 $(R^{k+1})_{ij} = 0, \forall i < j + (k + 1)$.

    对于任意 $1\leq i,j\leq n$, 有 $i < j + n$, 于是 $(R^n)_{ij} = 0$, 所以 $R^n = 0$.

    \item [15.] 习题18
    
    考查 $G = (D - \omega L)^{-1}\left((1-\omega)D + \omega U\right)$ 的特征值 $\lambda$, 有 :
    
    $\det(\lambda I - G) = \det\left((D - \omega L)^{-1}\right)\det\left(\lambda(D - \omega L) - (1-\omega)D - \omega U\right)=0$,

    即 $\det (\lambda (D - \omega L) - (1 - w)D - \omega U) = 0$. 下面验证 $B = \lambda (D - \omega L) - (1-\omega)D - \omega U$ 在 $|\lambda| \geq 1 $ 时是不可约弱对角占优矩阵:
    \begin{itemize}
        \item [(1)] $B$ 显然是不可约的,因为 $P^\top BP = (\lambda - 1 + \omega) P^\top DP -\lambda \omega P^\top LP - \omega P^\top U P$ 不可能具有 分块上三角矩阵的形式,这可以由 $A = D - L - U$ 的不可约性质得到。
        \item [(2)] $|B_{ii}| = |\lambda - 1 + \omega||a_{ii}|$. 当$j<i$时, $|B_{ij}| = |-\lambda \omega a_{ij}| = |\lambda\omega| |a_{ij}|$. 当 $j>i$ 时, $|B_{ij}| = |w||a_{ij}|$. 故 $|B_{ii}| - \sum_{j\ne i} |B_{ij}| \geq (|\lambda - 1 + \omega| - |\lambda\omega|) \sum_{j<i} |a_{ij}| + (|\lambda - 1 + \omega| - |\omega|)\sum_{j>i} |a_{ij}| \geq 0$. 由于 $A$ 是弱对角占优的,且 $|\lambda - 1 + \omega| \ne 0$, 所以至少存在一个 $i$ 使得上述不等式的第一个不等号不取等,所以 $B$ 是弱对角占优矩阵。 
    \end{itemize}
    所以对于 $|\lambda| \geq 1$, $B$ 是不可约的弱对角占优矩阵,根据定理8.6可知$B$ 非奇异,即$\det(B)\ne 0$. 故 $G$ 不存在模长大于等于 $1$ 的特征值,即 $\rho(G) < 1$, SOR迭代收敛。

    这里补充对 $|\lambda| \geq 1 \Rightarrow |\lambda - 1 + \omega| \geq |\lambda\omega| \geq |\omega|$ 的证明:

    $$\begin{array}{ll}|\lambda - 1 + \omega|^2 - |\lambda\omega|^2 &= (\lambda - 1 + \omega) (\bar\lambda - 1 + \omega) - \omega^2 \lambda\bar\lambda\\&=(1 - \omega^2)\lambda\bar\lambda - (1-\omega)(\lambda + \bar\lambda) + (1-\omega)^2\\&= (1-\omega)\left(\lambda\bar\lambda - \lambda - \bar\lambda + 1 + \omega(\lambda\bar\lambda - 1)\right)\\&=(1-\omega)\left((\lambda - 1)\overline{(\lambda - 1)} + \omega(\lambda\bar\lambda - 1)\right)\\&\geq 0\end{array}$$
    所以 $|\lambda - 1 + \omega| \geq |\lambda\omega| = |\lambda||\omega| \geq |\omega|$.

    \item [16.] 习题19
    
    \begin{itemize}
        \item [(1)] 对于任意 $x\in \mathbb{R}^{n\times 1}$, 有 $x^\top A^\top Ax = (Ax)^\top (Ax) \geq 0$. 当且仅当 $Ax = 0$, $x = A^{-1}Ax = 0$ 时取等,故 $A^\top A$ 是正定矩阵。其对称性显然。
        \item [(2)] 考虑 $A$ 的奇异值分解 $A = U\Lambda V$, 其中 $U$, $V$ 是正交矩阵,$\Lambda$ 是对角线矩阵,且由于 $A$ 非奇异,$\Lambda$ 对角线元素均非零。$A^\top A = V^\top \Lambda U^\top U\Lambda V = V^\top\Lambda^2 V$. 记 $\Lambda^2$ 中的最大值,最小值分别为 $\lambda_M,\lambda_m$. 易知 $A^\top AA^\top A = V^\top \Lambda^4 V$. 考虑到 $\mathtt{cond(A)_2} = \sqrt{\dfrac{\lambda_{\mathtt{max}}(A^\top A)}{\lambda_{\mathtt{min}}(A^\top A)}} = \sqrt{\dfrac{\lambda_{\mathtt{max}}(\Lambda^2)}{\lambda_{\mathtt{min}}(\Lambda^2)}}  = \sqrt{\dfrac{\lambda_M}{\lambda_m}}$, $\mathtt{cond(A^\top A)_2} = \sqrt{\dfrac{\lambda_{\mathtt{max}}(A^\top AA^\top A)}{\lambda_{\mathtt{min}}(A^\top AA^\top A)}} = \sqrt{\dfrac{\lambda_{\mathtt{max}}(\Lambda^4)}{\lambda_{\mathtt{min}}(\Lambda^4)}} = \dfrac{\lambda_M}{\lambda_m}$. 所以有 $\mathtt{cond}(A^\top A)_2 = \left(\mathtt{cond}(A)_2\right)^2$
    \end{itemize}

\end{itemize}
    
\end{spacing}

\end{document}